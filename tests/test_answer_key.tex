\documentclass{article}
\usepackage{amsmath}
\usepackage{amssymb}

\title{Sample Assignment - Answer Key}
\author{Professor Smith}
\date{\today}

\begin{document}
\maketitle

\section{Question 1}
Solution for quadratic equation $ax^2 + bx + c = 0$:

\begin{enumerate}
    \item[(a)] The discriminant is $\Delta = b^2 - 4ac$. This determines the nature of the roots.
    \item[(b)] Using the quadratic formula: $x = \frac{-b \pm \sqrt{b^2 - 4ac}}{2a}$
    \item[(c)] If $\Delta > 0$: two distinct real roots; If $\Delta = 0$: one repeated real root; If $\Delta < 0$: two complex conjugate roots
\end{enumerate}

\section{Question 2}
Solution for $f(x) = x^3 - 3x^2 + 2x$:

\begin{enumerate}
    \item[(a)] $f'(x) = 3x^2 - 6x + 2$
    \item[(b)] Setting $f'(x) = 0$: $3x^2 - 6x + 2 = 0$. Using quadratic formula: $x = \frac{6 \pm \sqrt{36-24}}{6} = \frac{6 \pm 2\sqrt{3}}{6} = 1 \pm \frac{\sqrt{3}}{3}$
    \item[(c)] Using second derivative test: $f''(x) = 6x - 6$. At $x = 1 - \frac{\sqrt{3}}{3}$: $f''(x) < 0$ (local max). At $x = 1 + \frac{\sqrt{3}}{3}$: $f''(x) > 0$ (local min).
\end{enumerate}

\section{Question 3}
Proof using L'Hôpital's rule:

Since both $\lim_{x \to 0} \sin x = 0$ and $\lim_{x \to 0} x = 0$, we have the indeterminate form $\frac{0}{0}$.

By L'Hôpital's rule: $\lim_{x \to 0} \frac{\sin x}{x} = \lim_{x \to 0} \frac{\cos x}{1} = \cos(0) = 1$

\end{document}
