\documentclass{article}
\usepackage{graphicx} % Required for inserting images

\title{homework 1}
\author{Evan Jiang}
\date{September 2025}

\begin{document}

\maketitle

\section{Q1.1}

(a) (1 point) For every state space graph there are costs on the edges such that
breadth-first search coincides with uniform-cost search.
True. If all of the costs are identical, uniform-cost search will be identical to
breadth-first search, as there is no lower cost edge to be preferenced by uniformcost search.
(b) (1 point) For every state space graph and costs on the edges there is a heuristic
function such that uniform-cost search coincides with A* search.
True. Using the cost of the edge as the heuristic function will make A* search
identical to uniform-cost search. This is not a consistent heuristic, but is admissible.
(c) (1 point) Depth-first search always expands at least as many nodes as A* search
with an admissible heuristic.
False. If we use the heuristic from (b) above (which is admissible), a long
path entirely consisting of edges with cost 1 (let’s say 5 edges) will be explored
partially before a shorter path consisting only of two edges with cost 2 begins
to be explored. If the shorter path happens to be the one chosen first by DFS,
the longer path will never be expanded at all, and DFS will expand fewer nodes
than A*.
(d) (1 point) Breadth-first search is complete even if zero step costs are allowed.
True. Breadth-first search disregards cost, so the cost values have no impact on
its completeness.
(e) (1 point) Assume that a rook can move on a chessboard any number of squares
in a straight line, vertically or horizontally, but cannot jump over other pieces.
Manhattan distance is an admissible heuristic for the problem of moving the
rook from square A to square B in the smallest number of moves.
1
False. An admissible heuristic must not be more than the actual number of
moves. A rook can move across an entire chessboard from one side to another in
a single move if the straight line is clear of other pieces. The Manhattan distance
of this move is more than one (seven, on a standard 8x8 chessboard), so it is not
an admissible heuristic.

\end{document}
